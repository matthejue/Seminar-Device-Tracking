\documentclass{report}

%!Tex Root = ../main.tex

% ┌────────────┐
% │ Formatting │
% └────────────┘
\usepackage[english]{babel}
\usepackage[export]{adjustbox} % use c, l, r for images
\usepackage{csquotes}
\usepackage[english]{babel}
\usepackage[parfill]{parskip}
% \usepackage[margin=1.5cm, headheight=0cm]{geometry}
\usepackage{geometry}
\usepackage{titlesec}
\usepackage{fix-cm}
\usepackage[usetoc]{titleref}

% ┌───────────────────┐
% │ Table of Contents │
% └───────────────────┘
\usepackage{titletoc}
\usepackage[nottoc]{tocbibind}

% ┌──────┐
% │ Math │
% └──────┘
\usepackage{amssymb} % for black triangleright, https://tex.stackexchange.com/questions/570303/use-blacktriangleright-as-itemize-label
\usepackage{amsmath}
\usepackage{mathtools} % for \mathclap and 
\usepackage{breqn}

% ┌────────┐
% │ Tables │
% └────────┘
\usepackage{tabularray}
\usepackage{wrapfig}
 % \UseTblrLibrary{diagbox}

% ┌────────┐
% │ Images │
% └────────┘
\usepackage{graphicx}
\usepackage{float} % for the letter H
% \graphicspath{figures/}
\usepackage[labelfont={color=PrimaryColor, bf}, textfont={it}]{caption}
\captionsetup{font=small}
\usepackage{subcaption}

% ┌──────────┐
% │ Diagrams │
% └──────────┘
\usepackage{tikz}
\usetikzlibrary{mindmap, shadows, backgrounds} % , calc
\usepackage{tikzit}
/home/areo/Documents/Studium/Semester_4_Master/Seminar/presentation/tikzit.sty

% ┌────────┐
% │ Citing │
% └────────┘
\usepackage[style=numeric]{biblatex}
\addbibresource{Seminar Device Tracking.bib}
\renewcommand*{\bibfont}{\footnotesize}

% \usepackage[nonumberlist, automake]{glossaries-extra}

% ┌─────────┐
% │ Itemize │
% └─────────┘
\usepackage{enumitem}
\setlist[itemize]{itemsep=0.0cm, parsep=0cm, topsep=0cm, partopsep=0cm, left=0cm}

% ┌────────────────────┐
% │ Code hightligthing │
% └────────────────────┘
\usepackage{minted}

% ┌───────────────────┐
% │ Header and Footer │
% └───────────────────┘
\usepackage{fancyhdr}

% ┌────────────────────────┐
% │ Latex Programming Help │
% └────────────────────────┘
\usepackage{etoolbox}
\usepackage{etoolbox}
\usepackage{xparse}
% https://tex.stackexchange.com/questions/358292/creating-a-subcounter-to-a-counter-i-created
\usepackage{chngcntr}
\usepackage{totcount}

% ┌───────┐
% │ Boxes │
% └───────┘
\usepackage{xcolor}
\usepackage{tcolorbox}
\tcbuselibrary{skins}
\tcbuselibrary{theorems}
\usetikzlibrary{patterns}
\tcbuselibrary{breakable}
\usepackage[bold=1]{xfakebold}
\tcbuselibrary{minted}

% ┌────────┐
% │ Colors │
% └────────┘
% \definecolor{PrimaryColor}{HTML}{344A9A}
% \definecolor{PrimaryColorDimmed}{HTML}{A4B1E0}
% \definecolor{SecondaryColor}{HTML}{006BB6}
% \definecolor{SecondaryColorDimmed}{HTML}{E5F0F8}
% \definecolor{SwitchColor}{named}{PrimaryColor}
% \colorlet{BoxColor}{gray!10!white}

\definecolor{PrimaryColor}{HTML}{ED870D}
\definecolor{PrimaryColorDimmed}{HTML}{EDC595}
\definecolor{SecondaryColor}{HTML}{F6AF3D}
\definecolor{SecondaryColorDimmed}{HTML}{F6DBB0}
\definecolor{SwitchColor}{named}{PrimaryColor}
\colorlet{BoxColor}{gray!10!white}

% ┌───────┐
% │ Links │
% └───────┘
% \usepackage[allbordercolors=PrimaryColor, pdfborder={0 0 .2}]{hyperref}
\usepackage[colorlinks=true, allcolors=PrimaryColor]{hyperref}

% ┌──────────┐
% │ Glossary │
% └──────────┘
% \usepackage[nonumberlist, automake]{glossaries-extra} % must come after hyperref

% ┌──────────────┐
% │ Pseudo Code  │
% └──────────────┘
\usepackage{pseudo}

% ┌────────────┐
% │ Misc Tools │
% └────────────┘
\usepackage{lipsum}

% ┌───────┐
% │ Fonts │
% └───────┘
\usepackage{fontspec}

%!Tex Root = ../main.tex

% ┌────────────┐
% │ Linebreaks │
% └────────────┘
% https://norwied.wordpress.com/2012/07/10/how-to-break-long-urls-in-bibtex/
\def\UrlBreaks{\do\/\do-\do\&\do.\do:}

% ┌───────┐
% │ Boxes │
% └───────┘
\newtcolorbox{titlebox}[1]{colback=PrimaryColorDimmed,colframe=PrimaryColor,arc=0.2cm,boxrule=0cm,frame hidden,left=0cm,right=0cm,top=0.1cm,toptitle=0.2cm,bottom=0.1cm,bottomtitle=0.2cm,boxsep=0cm,fonttitle=\bfseries,title={#1}}

% https://tex.stackexchange.com/questions/433256/inline-tcolorbox-with-rotated-title
\DeclareTotalTCBox{\points}{ s m }
{colback=PrimaryColorDimmed,boxrule=0cm,frame hidden,nobeforeafter,tcbox raise base,top=0mm,bottom=0mm,
	right=0mm,left=0mm,arc=0.1cm,boxsep=0.1cm}
{#2}

\DeclareTotalTCBox{\inlinebox}{ s m }
{verbatim,colback=PrimaryColorDimmed,colframe=PrimaryColor,nobeforeafter,tcbox raise base,top=0mm,bottom=0mm,
	right=0mm,left=0mm,arc=0.1cm,boxsep=0.1cm}
{\IfBooleanTF{#1}%
	{\textcolor{PrimaryColor}{\setBold >\enspace\ignorespaces}#2}%
	{#2}}

% \newtcbox{\inlineboxtwo}{capture=minipage,tcbox raise base,breakable,colback=PrimaryColorDimmed,colframe=PrimaryColor,nobeforeafter,top=0mm,bottom=0mm,
% 	right=0mm,left=0mm,arc=0.1cm,boxsep=0.1cm,fontupper=\ttfamily}

\DeclareTotalTCBox{\key}{ m }
{verbatim,colback=PrimaryColorDimmed,colframe=PrimaryColor,nobeforeafter,tcbox raise base,top=0mm,bottom=0mm,
	right=0mm,left=0mm,arc=0.1cm,boxsep=0.1cm}
{$\mathtt{#1}$}

\newtcolorbox{sidenote}{enhanced,boxrule=0cm,frame hidden,arc=0.2cm,title=Sidenote,attach boxed title to top text left={yshift=-0.2cm},colback=PrimaryColorDimmed,boxed title style={boxrule=0cm,frame hidden,colback=PrimaryColor,arc=0.1cm},fonttitle=\bfseries,
	after title={\hspace{0.2cm}\includegraphics[height=3mm]{./figures/lupe.png}}
}
% drop fuzzy shadow,

% https://tex.stackexchange.com/questions/210928/tcolorbox-how-do-i-align-on-the-baseline-of-the-title
\newtcblisting{terminal}{
	enhanced,box align=top,colframe=PrimaryColor,colback=PrimaryColorDimmed,hbox,arc=0.2cm,bottom=0.1cm,top=0.1cm,left=0.1cm,right=0.1cm,boxrule=0.05cm,listing only,minted language=text,listing engine=minted,minted options={escapeinside=||,autogobble}
}

\newtcblisting{terminal2}{
	enhanced,box align=top,colframe=PrimaryColor,colback=PrimaryColorDimmed,hbox,arc=0.2cm,bottom=0.1cm,top=0.1cm,left=0.1cm,right=0.1cm,boxrule=0.05cm,listing only,minted language=text,listing engine=minted,minted options={escapeinside=!!,autogobble}
}

\DeclareTCBInputListing{\file}{o m}{
	listing file={#2},enhanced,colframe=PrimaryColor,colback=PrimaryColorDimmed,hbox,fonttitle=\bfseries\tiny,halign title=center,arc=0.2cm,boxrule=0.05cm,bottom=0.1cm,top=0.1cm,left=0.1cm,right=0.1cm,listing only,listing engine=minted,#1}

\DeclareTCBListing{dfile}{o}{
	enhanced,colframe=PrimaryColor,colback=PrimaryColorDimmed,hbox,fonttitle=\bfseries\tiny,halign title=center,arc=0.2cm,boxrule=0.05cm,bottom=0.2cm,top=0.1cm,left=0.1cm,right=0.1cm,listing only,listing engine=minted,#1}

% https://tex.stackexchange.com/questions/593218/nested-inline-math-for-new-command-with-argument
\newcommand{\prompt}{\textcolor{PrimaryColor}{\setBold >\;\ignorespaces}}

\DeclareTCBInputListing{\codebox}{ s o m }{listing file={#3},
	enhanced,colframe=PrimaryColor,colback=BoxColor,IfBooleanTF={#1}{colframe=SecondaryColor}{colframe=PrimaryColor},fonttitle=\bfseries\small,#2,listing only,halign title=center,drop fuzzy shadow,arc=0,2cm,boxrule=0.05cm,bottom=0.1cm,top=0.1cm,left=0.1cm,right=0.1cm,listing engine=minted}
% , sharpish corners

\DeclareTCBListing{dcodebox}{ s o }{
	enhanced,colframe=PrimaryColor,IfBooleanTF={#1}{colframe=SecondaryColor,colback=SecondaryColorDimmed}{colframe=PrimaryColor,colback=PrimaryColorDimmed},hbox,fonttitle=\bfseries\small,#2,listing only,halign title=center,drop fuzzy shadow,arc=0.2cm,boxrule=0.05cm,bottom=0.1cm,top=0.1cm,left=0.1cm,right=0.1cm,listing engine=minted}
% , sharpish corners

\DeclareTCBInputListing{\file}{o m}{
	listing file={#2},enhanced,colframe=PrimaryColor,colback=PrimaryColorDimmed,hbox,,fonttitle=\bfseries,halign title=center,arc=0.2cm,bottom=0.2cm,top=0.1cm,left=0.1cm,right=0.1cm,boxrule=0.5mm,listing only,listing engine=minted,#1}

\DeclareTCBListing{dfile}{o}{
	enhanced,colframe=PrimaryColor,colback=PrimaryColorDimmed,hbox,,fonttitle=\bfseries,halign title=center,arc=0.2cm,bottom=0.2cm,top=0.1cm,left=0.1cm,right=0.1cm,boxrule=0.5mm,listing only,listing engine=minted,#1}

\newtcbinputlisting{\numberedcodebox}[2][]{
	listing file={#2}, enhanced, colframe=PrimaryColor,colback=BoxColor,fonttitle=\bfseries\small,#1,listing only,halign title=center,arc=0.2cm,boxrule=0.05cm,bottom=0.1cm,top=0.1cm,left=0.5cm,right=0.1cm,listing engine=minted,overlay={\begin{tcbclipinterior}\fill[PrimaryColorDimmed] (frame.south west) rectangle ([xshift=0.5cm]frame.north west);\end{tcbclipinterior}}
}

\newtcblisting{dnumberedcodebox}[1][]{
	enhanced, colframe=PrimaryColor,colback=PrimaryColorDimmed,fonttitle=\bfseries\tiny,#1,listing only,halign title=center,arc=0.2cm,boxrule=0.05cm,bottom=0.1cm,top=0.1cm,left=0.5cm,right=0.1cm,listing engine=minted,overlay={\begin{tcbclipinterior}\fill[PrimaryColor] (frame.south west) rectangle ([xshift=0.5cm]frame.north west);\end{tcbclipinterior}}
}

% https://tex.stackexchange.com/questions/585582/inside-of-a-newtcbinputlisting-how-can-i-change-the-color-of-the-line-number-as
\renewcommand{\theFancyVerbLine}{\sffamily
	\textcolor{white}{\tiny
		\oldstylenums{\arabic{FancyVerbLine}}}}

% ┌──────────────┐
% │ Bibliography │
% └──────────────┘
% \defbibenvironment{bibliography}
% {\itemize}
% {\enditemize}
% {\item}

% ┌──────────┐
% │ Commands │
% └──────────┘
\newtotcounter{exercisecounter}
\newtotcounter{exercisecounterdec}
\setcounter{exercisecounter}{1}
\newtotcounter{points}
\setcounter{points}{0}
\NewDocumentCommand\exercise{ o o }{
	\vspace{0.5cm}
	\IfValueTF{#1}
	{\textcolor{PrimaryColor}{\bfseries Exercise \theexercisecounter :} #1}
	{\textcolor{PrimaryColor}{\bfseries Exercise \theexercisecounter}}
	\stepcounter{exercisecounter}
	\IfValueT{#2}{
		\hfill\raggedleft\points{\qquad/#2 Subexercises}\ignorespaces\raggedright
		\addtocounter{points}{#2}
	}
}

% ┌──────────────────┐
% │ Case distinction │
% └──────────────────┘
% \newtoggle{whatever}
% \toggletrue{whatever}
% \togglefalse{whatever}
% \newcommand{\lpathx}[1]{\iftoggle{absolute}{/home/areo/Documents/Studium/Summaries/x/}{./}#1}

% ┌───────┐
% │ Share │
% └───────┘
\newcounter{algorithm}
\setcounter{algorithm}{0}
\newtcbtheorem[use counter=algorithm]{algorithm}{\color{PrimaryColor}Algorithm}{pseudo/ruled}{alg}


\newcommand{\cmt}[1]{\textcolor[gray]{0.7}{\textnormal{#1}}}

%!Tex Root = ../main.tex

% ┌────────────┐
% │ Formatting │
% └────────────┘
\setlength{\parskip}{0.4cm} % space between paragraphs, https://latexref.xyz/bs-par.html
\setlength\belowcaptionskip{-0.5cm} % https://stackoverflow.com/questions/2685152/gap-after-table-in-latex
% \mathversion{bold} % math formulas bold

% ┌───────────────────────┐
% │ Chapters and Sections │
% └───────────────────────┘
\titleformat{\chapter}[hang]
{\color{PrimaryColor}\normalfont\large\bfseries}{\thechapter}{0.5cm}{}

\titlespacing{\chapter}{0cm}{-0.5cm}{0cm}

\titleformat{\section}
{\color{PrimaryColor}\normalfont\normalsize\bfseries}
{\thesection}{0.5cm}{}
\titlespacing{\section}{0cm}{0.2cm}{0.2cm}
% \renewcommand{\thesection}{\arabic{section}}

\titleformat{\subsection}
{\color{PrimaryColor}\normalfont\normalsize\bfseries}
{\thesubsection}{0.5cm}{}
\titlespacing{\subsection}{0cm}{0.1cm}{0.1cm}

\makeatletter
\patchcmd{\chapter}{\if@openright\cleardoublepage\else\clearpage\fi}{}{}{}
\makeatother

% ┌───────────────────┐
% │ Table of Contents │
% └───────────────────┘
\titlecontents{chapter}[0cm]{\color{PrimaryColor}}{\thecontentslabel\quad}{\hspace{0cm}}{\titlerule*[0.5pc]{.}\contentspage}
\titlecontents{section}[0.5cm]{\color{PrimaryColor}}{\thecontentslabel\quad}{\hspace{1cm}}{\titlerule*[0.5pc]{.}\contentspage}
\titlecontents{subsection}[1cm]{\color{PrimaryColor}}{\thecontentslabel\quad}{\hspace{1cm}}{\titlerule*[0.5pc]{.}\contentspage}

% ┌───────────────────┐
% │ Header and Footer │
% └───────────────────┘
\renewcommand{\chaptermark}[1]{\markboth{\MakeUppercase{\thechapter. #1}}{}}
\fancypagestyle{plain}{%
	\fancyhf{}
	\renewcommand{\headrulewidth}{0.0mm}
	\fancyfoot[C]{\thepage}
}
\fancypagestyle{default}{%
	\fancyhf{}
	\fancyfootoffset{0.5cm}
	\fancyheadoffset{0.5cm}
  \fancyhead[L]{\color{PrimaryColor}\small\leftmark}
	\fancyhead[R]{\color{PrimaryColor}\small\rightmark}
	\renewcommand{\headrulewidth}{0.4mm}
  \renewcommand{\headrule}{\hbox to\headwidth{\color{PrimaryColor}\leaders\hrule height \headrulewidth\hfill}}
	\fancyfoot[C]{\thepage}
}

% ┌───────┐
% │ Fonts │
% └───────┘
% \newfontfamily\gyre{DejaVu Math TeX Gyre}
% colored bold
\newcommand\alert[1]{\textcolor{SecondaryColor}{\textbf{#1}}}
\newcommand\aalert[1]{\textcolor{PrimaryColor}{\textbf{#1}}}

% ┌──────────────┐
% │ Pseudo Code  │
% └──────────────┘
% \newcounter{algorithm}
% \setcounter{algorithm}{0}
% \newtcbtheorem[use counter=algorithm]{algorithm}{\color{SecondaryColor}Algorithm}{pseudo/ruled}{alg}
% \newcommand{\ma}[1]{$\mathcal{#1}$}
% \renewcommand{\tt}[1]{{\small\texttt{#1}}}

% ┌─────────┐
% │ Itemize │
% └─────────┘
\renewcommand{\labelitemi}{$\textcolor{SwitchColor}{\bullet}$}
\renewcommand{\labelitemii}{$\textcolor{SwitchColor}{\blacktriangleright}$}
\renewcommand{\labelitemiii}{$\textcolor{SwitchColor}{\blacksquare}$}

\renewcommand{\labelenumi}{\textbf{\textcolor{SwitchColor}{\theenumi.}}}
\renewcommand{\labelenumii}{\textbf{\textcolor{SwitchColor}{\theenumii.}}}
\renewcommand{\labelenumiii}{\textbf{\textcolor{SwitchColor}{\theenumiii.}}}

% ┌──────────┐
% │ Glossary │
% └──────────┘
\newcommand{\ditem}[2]{%
\item \hypertarget{#1}{\alert{#2}}:%
}

% \usepackage[nonumberlist, automake]{glossaries-extra}
% \newglossary[conv]{conv}{con}{cns}{}
% \makeglossaries
%
% \renewcommand*{\glsnamefont}[1]{\textcolor{PrimaryColor}{#1}}
% \renewcommand*{\glossarysection}[2][]{}


\begin{document}

\fontsize{8pt}{9pt}\selectfont
% \fontsize{10pt}{11pt}\selectfont
\newgeometry{margin=2cm, bottom=2cm, top=2cm}

\begin{titlepage}
  \vspace{1cm}
  \center
  \textsc{\LARGE Albert Ludwig's University Freiburg}\\[0.5cm]
  \textsc{\Large Technical Faculty}\\[2.0cm]

  \vspace{1cm}

	\begin{titlebox}{\center \huge \bfseries Device Tracking via Linux’s New TCP Source Port Selection Algorithm}
		\centering
		% \bfseries \Large New Methods in Hard Disk Encryption\cite{fruhwirth2005new} and XTS Cipher Mode\cite{TweakingTweakableAES}
		\bfseries \Large Paper by Moshe Kol, Amit Klein and Yossi Gilad\cite{kol2022devicetrackinglinuxsnew}
	\end{titlebox}

  \textsc{\large Report}\\
  \rule{\linewidth}{0.1mm}

  \vspace{3cm}

  \begin{minipage}[t]{0.45\textwidth}
    \begin{flushleft} \large
      \emph{Author:}\\
      Jürgen Mattheis
    \end{flushleft}
  \end{minipage}
  ~
  \begin{minipage}[t]{0.45\textwidth}
    \begin{flushright} \large
      \emph{Supervising Professor:}\\
      Prof. Dr. Christian Schindelhauer\\[0.64cm]
      \emph{Supervising Assistants:}\\
      Sneha Mohanty,\\
      Saptadi Nugroho,\\
      Joan Bordoy,\\
      Wenxin Xiong
    \end{flushright}
  \end{minipage}

  \vspace{7cm}
  \rule{12cm}{0.1mm}\\[0.25cm]
  \large{Seminar at the chair for Computer Networks and Telematics}
\end{titlepage}


\tableofcontents
\thispagestyle{empty}
\clearpage
\pagestyle{plain}
\pagenumbering{Roman}

\listoffigures
% \newpage
% \listoftables

\clearpage
\fancypagestyle{plain}{%
	\fancyhf{}
	\fancyfootoffset{0.5cm}
	\fancyheadoffset{0.5cm}
	\fancyhead[L]{\color{PrimaryColor}\small\leftmark}
	\fancyhead[R]{\color{PrimaryColor}\small\rightmark}
	\renewcommand{\headrulewidth}{0.4mm}
	\renewcommand{\headrule}{\hbox to\headwidth{\color{PrimaryColor}\leaders\hrule height \headrulewidth\hfill}}
	\fancyfoot[C]{\thepage}
}
\pagenumbering{arabic}
\pagestyle{default}

\chapter{Introduction}
\label{sec:introduction}

In 2018, the \alert{Meltdown} attack caused widespread concern by revealing how speculative execution in modern CPUs could be exploited to read sensitive data from privileged memory that should have been inaccessible to user processes. In a similar vein, researchers Moshe Kol, Amit Klein, and Yossi Gilad from the Hebrew University of Jerusalem discovered that an older version of Linux’s TCP source port selection algorithm could be exploited to create persistent device finterprints that survive browser privacy modes and network changes. % and VPNs

Like Meltdown, their attack targets \alert{security} (not safety: it does not crash systems or corrupt data), it violates confidentiality by leaking internal state. Meltdown extracts data from CPU caches, while this technique exploits patterns of TCP source port allocations determined by a secret per-device key, allowing an attacker to derive a unique \alert{device fingerprint} without revealing the key itself.

\section{About this report}
\label{sec:goal_of_this_report}

The goal of this report is to provide insight into device tracking techniques, specifically by explaining the method described in the paper\cite{kol2022devicetrackinglinuxsnew}. This report details how the tracking method works and explains the proposed countermeasures and the final patch that was integrated into the Linux kernel to mitigate this vulnerability.

\alert{It should be emphasized that this is just a report, it is just explaining, connecting and summarizing different results of other people's research and doesn't contain own research or claim this to be the results of the research of the author of this report}. 


%Furthemore a reader of this report should make sure to read the appendix section \ref{sec:conventions} to be sure to understand the conventions used within this report. %But the author of this report supposes that the conventions should be intuitively clear from the context.

\section{Online browser-based device tracking}
\label{sec:Online browser-based device tracking}

\section{Source Port Selection}
\label{sec:source port selection}

\subsection{Double-Hash Port Selection Algorithm (DHPS)}
\label{sec:double-hash port selection algorithm}

\begin{center}
	% \begin{adjustbox}{scale=1}
		\begin{minipage}[t]{0.5\textwidth}
			\vspace{0cm}

			%!Tex Root = ../main.tex

\begin{algorithm}{\pr{DHPS Source Port Selection}}{\thetcbcounter}
	\begin{pseudo}[indent-mark,kw,hl-warn=false]
		procedure \cn{SelectEphemeralPort} \\+
		\tt{num\_ephemeral} $\leftarrow$ \tt{max\_ephemeral}−\tt{min\_ephemeral}+\tn{1} \\
		\tt{offset} $\leftarrow$ \cn{$F_{K_1}$}($IP_{SRC}$,$IP_{DST}$,$PORT_{DST}$) \\
		\tt{index} $\leftarrow$ \cn{$G_{K_2}$}($IP_{SRC}$,$IP_{DST}$,$PORT_{DST}$) \\
		\tt{count} $\leftarrow$ \tt{num\_ephemeral} \\
		repeat \\+
		\tt{port} $\leftarrow$ \tt{min\_ephemeral}+\\+
		((\tt{offset}+\tt{table[index]}) mod \tt{num\_ephemeral}) \\-
		\tt{table[index]} $\leftarrow$ \tt{table[index]} +\tn{1} \\
		if \cn{CheckSuitablePort}(\tt{port}) then \\+
		return \tt{port} \\-
		\tt{count} $\leftarrow$ \tt{count} − \tn{1} \\-
		until \tt{count} = \tn{0} \\
		return \cn{Error}\\
	\end{pseudo}
\end{algorithm}

		\end{minipage}
	% \end{adjustbox}
\end{center}


\vspace{0.5cm}
\chapter{Attack}
\label{sec:attack}

\section{Technical details}
\label{sec:technical details}

\section{Attack model} % (fold)
\label{sec:attack model}

\section{Device ID}
\label{sec:device id}

\section{Limitations}
\label{sec:limitations}

\section{Attack Overview}

\subsection{Phase 1}
\label{sec:phase 1}

\begin{center}
  % \begin{adjustbox}{scale=0.6}
    \begin{minipage}[t]{0.5\textwidth}
      \vspace{0cm}
      
      %!Tex Root = ../main.tex

\begin{algorithm}{\pr{Finding Attacker 3-Tuple per Cell (Phase 1)}}{\thetcbcounter}
	\begin{pseudo}[indent-mark,kw,hl-warn=false]
    procedure \cn{SendBurst}(\tt{X})\\+
    for all \tt{x} $\in$ \tt{X} do\\+
\cn{AttemptConnectTCP}(\tt{x})\\--
procedure \cn{GetSourcePorts}(\tt{U})\\+
\cn{SendBurst}(\tt{U})\\
\tt{R} $\leftarrow$ \cn{ReceiveAttackerTupleToPortMap}()\\
\cmt{// R = $\{(\tt{IP}_{SRC}$,$\tt{IP}_{DST}$,$\tt{PORT}_{DST}$) $\mapsto \tt{PORT}_{SRC}\}$}\\
\cmt{// (obtained from the tracking server)}\\
return \tt{R}\\-
procedure \cn{Phase1}\\+
\tt{S'} $\leftarrow\emptyset$\\
while $|\tt{S'}| < \tt{T}$ do\\+
$\tt{S}_i$ $\leftarrow$ \cn{GetNewExternalDestinations}()\\
\cmt{// $\forall_{j<i}(\tt{S}_i \cap \tt{S}_j = \emptyset)$}\\
\tt{P} $\leftarrow$ \cn{GetSroucePorts}($\tt{S}_i$) \cmt{// 1st burst}\\
\cn{SendBurst}(S') \cmt{// 2nd burst}\\
\tt{P'} $\leftarrow$ \cn{GetSourcePorts}($\tt{S}_i$) \cmt{// 3rd burst}\\
\tt{S'} $\leftarrow \tt{S'} \cup \{x\,|\,\tt{P'}(x)−\tt{P}(x)=1\}$\\
\cmt{// $\tt{V}_i = \{\tt{x}\,|\,\tt{P'}(x)-\tt{P}(x) = 1\}$}\\-
return \tt{S'}
	\end{pseudo}
\end{algorithm}

    \end{minipage}
  % \end{adjustbox}
\end{center}


\subsection{Phase 2}
\label{sec:phase 2}

\subsection{Terminating with an Accurate ID}
\label{sec:terminating with an accurate id}

\vspace{0.5cm}
\chapter{Countermeasures}
\label{sec:countermeasures}

\section{Possible solutions}
\label{sec:possible solutions}

\section{Final solution}
\label{sec:key_hierarchies}

\vspace{0.5cm}
\chapter{Related Work}
\label{sec:related work}

\vspace{0.5cm}
\chapter{Appendix}
\label{sec:appendix}
\pagenumbering{Alph}

\section{Conventions}
\label{sec:conventions}

In this report certain topics weren't covered and certain conventions were followed, all this might-be-useful-before-reading-the-report information is listed here:
\begin{itemize}
	\item asdf
\end{itemize}

\vspace{0.5cm}
\chapter{Bibliography}
\label{sec:bibliography}

\printbibliography[heading=none]

% \huge\textcolor[HTML]{0563aa}{$\mathcal{E}$}

\end{document}
